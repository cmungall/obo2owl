
% FOR SUBMISSION TO OWLED
% http://www.webont.org/owled/2009/

% This is LLNCS.DEM the demonstration file of
% the LaTeX macro package from Springer-Verlag
% for Lecture Notes in Computer Science,
% version 2.3 for LaTeX2e
%
\documentclass{llncs}
%

\def\PE{\pr{Physical Entity}}
\def\Pr{\pr{Process}}
\def\Sys{\pr{System}}
\def\Snap{\pr{Snapshot}}
\def\State{\pr{State}}
\def\Property{\pr{Property}}
\def\Tp{\pr{Timepoint}}
\def\pred{\pr{Predicate}}
\def\lacksPart{lacksPart}
\def\capableOf{capableOf}

\newtheorem{expr}{}[section]
\setcounter{expr}{0}
\renewcommand{\theexpr}{A-\arabic{expr}}

\newtheorem{exprule}{}[section]
\setcounter{exprule}{0}
\renewcommand{\theexpr}{R-\arabic{exprule}}

\newcommand{\mod}[1]{\textbf{#1}}


\def\correspondingauthor{$^*$}
\def\@corresponding{\footnotesize\correspondingauthor Corresponding author} 

\def\address#1{ \def\@address{\begin{hi}\footnotesize#1\end{hi}}}
\def\iid(#1){\hi$^#1$}

%%%%%%%%%%%%%%%%%%%%%%%%%%%%%%%%%%%%%%%%%%%%%%%%%%%%%%%%%%%%%%%
%%                    ** mymacros.tex **                     %%
%%%%%%%%%%%%%%%%%%%%%%%%%%%%%%%%%%%%%%%%%%%%%%%%%%%%%%%%%%%%%%%


\typeout{** loading mymacros.tex **}

\def\h{\hbox}

\newcommand{\mathcom}[3]{ \newcommand{#1}[#2]{\mbox{$#3$}}}
\newcommand{\remathcom}[3]{ \renewcommand{#1}[#2]{\mbox{$#3$}}}

\newcommand{\mdef}[2]{ \newcommand{#1}{\mbox{$#2$}} }
 
\mathcom{\y}{0}{\ \vdash\ }               % yeilds 
\mathcom{\ys}{0}{\vdash_{\cal S}}         % yeilds sub S
\mathcom{\yt}{0}{\vdash_{\theta}}         % yeilds sub theta

\mathcom{\absurd}{0}{\mathbf{f}}                 % absurdity
\mathcom{\imp}{0}{\ \rightarrow\ }            % implication arrow
\mathcom{\rimp}{0}{\ \leftarrow\ }            % implication arrow

\mathcom{\con}{0}{\ \wedge\ }                 % conjunction
\mathcom{\dis}{0}{\ \vee\ }                   % disjunction
\mathcom{\n}{0}{\neg}                     % negation
\mathcom{\dimp}{0}{\ \leftrightarrow\ }       % mat equiv

\mathcom{\corresponds}{0}{\ \Lleftarrow\! \! \Rrightarrow\ }


%%\mathcom{\th}{0}{\theta}                  % theta


\mathcom{\A}{0}{\forall}                  % universal quantifier
\mathcom{\E}{0}{\exists}                  % existential quant

\mathcom{\Dec}{1}{{\cal D}(#1)}           % D(#1)

\mathcom{\elt}{0}{\in}

\mathcom{\tuple}{1}{\langle #1 \rangle}

% \mathcom{\equivdef}{0}{\equiv_{\mbox{\em \tiny def}}}
% \mathcom{\eqdef}{0}{=_{\mbox{\em \tiny def}}}
\def\equivdef{\mathrel{\ \equiv_{\mbox{\em \tiny
def}}\ }}

\def\eqdef{\mathrel{\ =_{\mbox{\em \tiny def}}\ }}


% \newtheorem{Rule}{Rule}
% \newtheorem{Lemma}{Lemma}
% \newtheorem{Theorem}{Theorem}


% Redeclaration of symbols for intuitionistic logic
% see Diller p.126
\def\ineg{\mathop\sim}
\def\iimp{\mathbin\Rightarrow}
\def\iiff{\mathbin\Leftrightarrow}
% new symbols from amssymb
%\def\iimp{\mathbin\rightarrowtail}
%\def\iiff{\mathbin\leftrightsquigarrow}

\def\icon{\mathbin\curlywedge}
\def\idis{\mathbin\curlyvee}

%Definitions for RCC inserts
\newcommand{\notch}{\Longrightarrow\kern-16pt{/}\ \ \ }
\newcommand{\ch}{\Longrightarrow}
\newcommand{\pr}[1]{\mbox{\sf #1}}

% RCC relations
\mathcom{\C}{0}{\pr{C}}
\remathcom{\P}{0}{\pr{P}}
\remathcom{\O}{0}{\pr{O}}
\mathcom{\PO}{0}{\pr{PO}}
\mathcom{\PP}{0}{\pr{PP}}
\mathcom{\NTPP}{0}{\pr{NTPP}}
\mathcom{\NTPPI}{0}{\pr{NTPP}^{-1}}
\mathcom{\TPP}{0}{\pr{TPP}}
\mathcom{\TPPI}{0}{\pr{TPP}^{-1}}
\mathcom{\TP}{0}{\pr{TP}}
\mathcom{\NTP}{0}{\pr{NTP}}
\mathcom{\NTPi}{0}{\pr{NTPi}}
\mathcom{\EC}{0}{\pr{EC}}
\mathcom{\DC}{0}{\pr{DC}}
\mathcom{\DR}{0}{\pr{DR}}
\mathcom{\EQ}{0}{\pr{EQ}}
\mathcom{\CG}{0}{\pr{CG}}

\mathcom{\Pin}{0}{\pr{Pi}} % \Pi is Greek letter
\mathcom{\PPi}{0}{\pr{PPi}}
\mathcom{\NTPPi}{0}{\pr{NTPPi}}

\mathcom{\OP}{0}{\pr{OP}}

\mathcom{\conv}{0}{\pr{conv}}
\mathcom{\rsum}{0}{\pr{sum}}
\mathcom{\rdiff}{0}{\pr{diff}}
\mathcom{\rcompl}{0}{\pr{compl}}
\mathcom{\rprod}{0}{\pr{prod}}
\mathcom{\cp}{0}{\pr{cp}}
\mathcom{\Us}{0}{\pr{Us}}
\mathcom{\us}{0}{\pr{u}}
\mathcom{\NULL}{0}{\pr{NULL}}
\mathcom{\rnull}{0}{\emptyset}
\mathcom{\CONV}{0}{\pr{CONV}}
\mathcom{\CON}{0}{\pr{CON}}

% Adaptations for new latex compatibility
\renewcommand{\Box}{\square}
\def\cal{\mathcal}

% meta language in definitions etc
\mathcom{\ml}{1}{\;\;\;\; \mbox{#1} \;\;\;\;}


%-- TOPLOG SYMBOLS

\mathcom{\Co}{0}{{\cal C}_{0} \ }
\mathcom{\CoX}{0}{{\cal C}_{0}}
\mathcom{\Cop}{0}{{\cal C}_{0}^+ \ }
\mathcom{\CopX}{0}{{\cal C}_{0}^+}
\mathcom{\mcop}{0}{\models_{C_o^+}}

\mathcom{\Lop}{0}{{\cal L}_{0}^+ \ }
\mathcom{\LopX}{0}{{\cal L}_{0}^+}
\def\mlo{\models_{{\mathcal  L}_0}}
\def\mlop{\models_{{\mathcal  L}_0^+}}

\mathcom{\Ci}{0}{{\cal C}_{1} \ }
\mathcom{\CiX}{0}{{\cal C}_{1}}

\mathcom{\Io}{0}{{\cal I}_{0} \ }
\mathcom{\IoX}{0}{{\cal I}_{0}}
\mathcom{\Iop}{0}{{\cal I}_{0}^+ \ }
\mathcom{\IopX}{0}{{\cal I}_{0}^+}

%%\mathcom{\inter}{1}{\langle #1 \rangle}
\mathcom{\inter}{1}{i( #1 )}

\newfont{\myssfont}{cmss12}

\mathcom{\Maps}{2}{\, _{#1}\!\!\!\rightleftharpoons^{#2} \ }

\mathcom{\CotoST}{0}{\Maps{C_0}{ST}}
\mathcom{\CtoST}{0}{\Maps{C\,}{ST}}
\mathcom{\IotoST}{0}{\Maps{I_0}{ST}}

\mathcom{\uni}{0}{\cal U \, }
\mathcom{\uniX}{0}{\cal U}
\mathcom{\U}{0}{\cal U}

\mathcom{\eu}{0}{\,=\,\uni}

\mathcom{\ol}{1}{\overline{#1}}

\mathcom{\Lsse}{0}{{\cal L}_{sse}}
\mathcom{\Lssei}{0}{{\cal L}_{ssei}}

\mathcom{\Luse}{0}{{\cal L}_{use}}
\mathcom{\Lusei}{0}{{\cal L}_{usei}}

\mathcom{\disdots}{0}{\dis\!\!\!\ldots\!\!\dis}
\mathcom{\condots}{0}{\con\ldots\con}





%% Miscelleneous


\def\HA#1#2{\pr{Holds-At}(#1,#2)}
\def\col{\!:\!}

% Enclose arguments in square double brackets to
% refer to semantic value of an expression
\def\semvalue#1{[\mkern-3mu[#1]\mkern-3mu]}

\def\forces{\mathop{\setbox0=\hbox{$\vdash$}\ \rule{0.04em}{1\ht0}\mkern1.5mu\box0}}

\mathcom{\hence}{0}{\Longrightarrow \hspace{2em}}

\def\sliderule{\centerline{\rule{4in}{1mm}}}

\def\Cbox{\mathop{\square\mkern-10mu
                   \mbox{\raise0.45ex\hbox{\tiny\sf C}}
                  \mkern2mu
                 }
         }

\def\letterbox#1{\mathop{\square\mkern-10mu
                   \mbox{\raise0.2ex\hbox{\small\sf #1}}
                  \mkern2mu
                 }
         }

%\def\tBox{\letterbox{\lower0.4ex\hbox{t}}}
\def\sBox{\letterbox{s}}
\def\tBox{\letterbox{t}}

\def\overstrike#1#2{\mathop{%
               \setbox1=\hbox{$#1$}%
               \setbox2=\hbox{$#2$}%
               \ifdim 1\wd1<1\wd2 % 
                    { \rlap{\mbox{\kern0.5\wd2\kern-0.5\wd1 \box1}} \box2 }
              \else {   \rlap{\mbox{$#1$}}
                        \rlap{\mbox{\kern0.5\wd1\kern-0.5\wd2 \box2}}
                        \phantom{#1} } 
              \fi }}

\def\rbox{\boxminus}
\def\rdia{\overstrike{\Diamond}{-}}
\def\cdia{\overstrike{\Diamond}{\rule{0.04em}{1.5ex}}}
\def\cbox{\inbox{\rule{0.04em}{1.6ex}}}
\def\cbox{\mathop{\setbox0=\hbox{$\square$}
                  \overstrike{\square}{\hbox{\rule{0.04em}{1\ht0}}}
             }}

\def\diaplus{\overstrike{\Diamond}{+}}




%% Make a modal box with a symbol centred inside
\def\inbox#1{\mathop{\setbox0=\hbox{$\square$}%
                      \setbox1=\hbox{#1}%
                      \rlap{\hbox{$\square$}}
                      \rlap{\mbox{\kern0.5\wd0\kern-0.5\wd1%
                                  \box1}}%
                      \phantom{\square}}}

\def\ibox{\mathop{\square\mkern-10mu
                   \mbox{\raise0.45ex\hbox{\tiny\em i}}
                  \mkern7mu
                 }
         }
\def\cdiamond{\mathop{\Diamond\mkern-12.7mu
                   \mbox{\raise0.4ex\hbox{\tiny\em c}}
                  \mkern7mu
                 }
         }


\def\s5box{\mathop{\square\mkern-10mu
                   \mbox{\raise0.45ex\hbox{\tiny 5}}
                  \mkern7mu
                 }
         }

\def\Fbox{\mathop{\square\mkern-11mu
                   \mbox{\raise0.45ex\hbox{\tiny \bf F}}
                  \mkern2mu
                 }
         }

\def\Box{\mathop\square}
\def\Diamond{\mathop\lozenge}

\def\bigI{\mathop{%
\hbox{%
%%
\def\thickness{1pt}%
\def\width{6pt}%
\def\height{0.6em}%
\def\gap{0.15em}%
%%
\dimen0=\thickness%
\divide\dimen0 by 2%
\dimen1=\width%
\divide\dimen1 by 2%
%%
\kern\gap%
\rlap{\vrule width\width height\thickness depth0pt}%
\rlap{\kern\dimen1 \kern-\dimen0%
\vrule width\thickness height\height depth0pt}%
\raise\height\hbox{\vrule width\width height\thickness depth0pt}%
\kern\gap}}}

\def\mconv{\mathop{% 
      \mbox{\raise0.35ex\hbox{$\scriptstyle\bigcirc$}}
                  }
          }
\def\mpconv{\circleddash}



% \def\putat#1#2#3{\setbox1=\hbox{#3}%
% \rule{0pt}{0pt}%
% \vspace*{#2}%
% \hspace*{#1}%
% \rule{0pt}{0pt}%
% \hbox to 0em{\rlap{\vtop to 0ex{\hbox to 0em{#3}}}}%
% \hspace*{-#1}\vspace*{-#2}}
% %\rule{0pt}{0pt}
% %\vspace*{-\ht1}\rule{0pt}{0pt}
% %\vspace*{-\ht1}

\def\putat#1#2#3{\rlap{\hbox{%
\kern#1% 
\vtop{\hbox{}%
\hbox{\vbox to #2{}} \hbox{#3}}}}}

% use putat within \hbox
% Objects will be located relative to baseline
% at start of the \hbox
% eg:
% \hbox{
% \putat{2in}{2in}{xxxxx}%
% \putat{2in}{2in}{00000}%
% \putat{5in}{3in}{Z}%
% \putat{2in}{2in}{0}%
% \putat{5in}{4in}{x}%
% }


%\long\def\putat#1#2#3{




\def\ri{\mathclose{\raise1ex\hbox{$\smile$}}}

\def\eqtag#1{\eqno ({\bf #1})}

\def\mc:#1{\hbox{${\mathcal{#1}}$}}
\def\mb#1{{\mathbf{#1}}}
 
\def\ifff{\qquad \hbox{if and only if} \qquad}

%%\def\d{\delta}

\def\QED{$\blacksquare$}

\def\epsrcgrant{GR/K65041}
\def\VUGgrant{GR/M56807}

\def\Boxx{\boxtimes}

%% Force a math symbol to go into scriptsize
\def\mscriptsize#1{\hbox{\scriptsize $#1$}}
\def\mlarge#1{\hbox{\large $#1$}}

\def\boxtheorem#1#2{~\\
\centerline{\fbox{
\parbox{5in}{
\centerline{\bf #1}
#2
}}}
\vspace{1ex}}

\def\proof#1#2{
\begin{quotation}
\noindent {\bf Proof of #1:}
#2
\QED
\end{quotation}
}

\def\proofpar{\\ \hspace*{1.5em}}


%%%% ENVIRONMENTS

%% specify my list parameters
%% These can be overridden by declarations within
%% a document.

\def\mytopsep{1ex}
\def\mypartopsep{0ex}
\def\myitemsep{0.3ex}
\def\myparsep{0ex}
\def\mylistparindent{2em}
\def\myleftmargin{4.5em}
\def\mylabelwidth{4.5em}
\def\mylabelsep{1em}
\def\myitemindent{0em}

\def\setmylistparams{%
             \setlength{\topsep}{\mytopsep}
             \setlength{\partopsep}{\mypartopsep}
             \setlength{\itemsep}{\myitemsep}
             \setlength{\parsep}{\myparsep}
             \setlength{\listparindent}{\mylistparindent}
             \setlength{\leftmargin}{\myleftmargin}
             \setlength{\labelwidth}{\mylabelwidth}
             \setlength{\labelsep}{\mylabelsep}
             \setlength{\itemindent}{\myitemindent}
        }

\def\clistbullet{$\bullet$}

\newcounter{tempvalue}
% A compact list making environment
\newenvironment{clist}
    { %\vspace{1ex}
      \begin{list}
            {\labelitemi}
            \setmylistparams
    }
    { \end{list}\addvspace{1ex} 
    }
% A compact list making environment
% with -- as the default item tag
\newenvironment{cdashlist}
    { %\vspace{1ex}
      \begin{list}
            {--}
            {\setlength{\topsep}{0.2ex}
             \setlength{\partopsep}{0ex}
             \setlength{\itemsep}{0.3ex}
             \setlength{\parsep}{0ex}
             \setlength{\listparindent}{2em}
            }
    }
    { \end{list}\addvspace{1ex} 
    }


% compact list with wide labels for item tags
\newenvironment{taglist}[1]
    { %\vspace{1ex}
      \begin{list}
            {$\bullet$}
            {\setlength{\topsep}{0.2ex}
             \setlength{\partopsep}{0ex}
             \setlength{\itemsep}{0.3ex}
             \setlength{\parsep}{0ex}
             \setlength{\listparindent}{2em}
             \setlength{\leftmargin}{#1}
             \setlength{\labelwidth}{#1}
             %\addtolength{\labelwidth}{-1em}
             \setlength{\labelsep}{0em}
            }
    }
    { \end{list}\addvspace{1ex} 
    }

\newcounter{cenumcount}
\newenvironment{cenum}
    { %\vspace{1ex}
      \begin{list}
            {\arabic{cenumcount}.}
            {\usecounter{cenumcount}
               \setmylistparams
            }
    }
    { \end{list}\addvspace{1ex} 
    }

\newcounter{romcount}
\newenvironment{romanenum}
          { 
            \begin{list}{\roman{romcount})~~}
                        {\usecounter{romcount}
                         \setmylistparams
                        }
          }
          { \end{list} }

\newcounter{alphcount}
\newenvironment{alphenum}
          { \begin{list}{\alph{alphcount})~~}
                        {\usecounter{alphcount}}}
          { \end{list} }


%% Labelled list environment
\newenvironment{llist}[1]{%
\newcounter{#1}
\begin{list}{({\bf #1\arabic{#1}})\hspace{1em}}{\usecounter{#1}}
}
{\end{list}}

%% Labelled list continued
%% (assumes counter already exists)
\newenvironment{llistcont}[1]{%
\setcounter{tempvalue}{\value{#1}}
\begin{list}{({\bf #1\arabic{#1}})\hspace{1em}}{%
\usecounter{#1}
\setcounter{#1}{\value{tempvalue}}
}
}
{\end{list}}

%%% compact versions of llist and llistcont
\newenvironment{cllist}[1]{%
\newcounter{#1}
\begin{list}{({\bf #1\arabic{#1}})}
             { \usecounter{#1}
               \setmylistparams
             }
}
{\end{list}}

%% Labelled list continued
%% (assumes counter already exists)
\newenvironment{cllistcont}[1]{%
\setcounter{tempvalue}{\value{#1}}
\begin{list}{({\bf #1\arabic{#1}})}{%
                 \usecounter{#1}
                 \setcounter{#1}{\value{tempvalue}}
                \setmylistparams
          }
}
{\end{list}}


\def\longdef{\long\def}

%% Define a labelled list type
%% #1 name of control sequence for the list type
%% #2 prefix for numbering
\def\llisttype#1#2{%
\newcounter{#2}
\expandafter\longdef\csname #1list\endcsname##1{%
\begin{cllistcont}{#2}
##1
\end{cllistcont}}
\expandafter\def\csname #1\endcsname##1{%
\begin{cllistcont}{#2}
\item
##1
\end{cllistcont}}
}

\def\litem#1#2{%
\begin{clist}
\item[({\bf #1})] 
#2
\end{clist}}

\newenvironment{chapabst}%
{\begin{quotation}\small\noindent }{\end{quotation}}

% Split line displayed formulae
% (not actually an environment)
\newcommand{\splitdismath}[4]{%
{\setlength{\arraycolsep}{0em}
\begin{eqnarray}
& #1 & #2 \nonumber \\
 &    & #3 
\label{#4}
\end{eqnarray}
}
}

% Displayed equation with tabbing

\newcommand{\tabmathn}[2]
{ \begin{equation}
\vbox{
\vspace{-2ex}
\begin{tabbing}
#1
\end{tabbing}
\vspace{-3.5ex}
}
\label{#2}
\end{equation}
}

\newcommand{\displaytab}[1]
{ \begin{displaymath}
\vbox{
\vspace{-2ex}
\begin{tabbing}
#1
\end{tabbing}
\vspace{-3.5ex}
}
\end{displaymath}
}

\def\myeq#1{$$\eqalign{#1}$$}


%%% Reinstate \eqalign macros removed by Lamport
\catcode`\@=11
\def\eqalign#1{\null \,\vcenter {\openup \jot \m@th 
\ialign {\strut \hfil $\displaystyle{##}$&$\displaystyle {{}##}$\hfil
 \crcr #1\crcr }}\,}

\def\eqalignno#1{\displ@y \tabskip \centering 
\halign to\displaywidth {\hfil $\@lign \displaystyle{##}$\tabskip \z@skip
 &$\@lign \displaystyle {{}##}$\hfil \tabskip \centering 
 &\llap {$\@lign ##$}\tabskip \z@skip \crcr 
 #1\crcr }}

\def\leqalignno#1{\displ@y \tabskip \centering
 \halign to\displaywidth {\hfil $\@lign \displaystyle {##}$\tabskip \z@skip
 &$\@lign \displaystyle {{}##}$\hfil \tabskip \centering
 &\kern -\displaywidth \rlap {$\@lign ##$}\tabskip \displaywidth \crcr 
 #1\crcr}}

%%% Define something like \bibliography but only records the
%%% bibfiles (does not input the bbl file)
\def\bibliographyfiles#1{%
  \if@filesw
    \immediate\write\@auxout{\string\bibdata{#1}}%
  \fi}

%%% Define a separate command to input the bbl file.
\def\bibliographytext{%
  \@input@{\jobname.bbl}}


%% Suppress printing of bibitems in thebibliography
%% This is for use with the `inlinebib' package,
%% which adds fullcitations where you give the \cite command.
%% Doesn't print note
%% Could add note by changing ##2 to ##2##3 in the \bibcite argument
%% But output not quite right.
\def\suppressendbib{%
\long\def\@bibitem##1 ##2 \note ##3 \short ##4 \end{\par\if@filesw
        {\def\protect####1{\string ####1\space}%
         \let\newblock\@empty
         \immediate\write\@auxout{\string
                \bibcite{##1}{\string\@bibcall{##1}{##2}{##4}}}}\fi
        }
\def\thebibliography##1{}
}


%% Change indent space allocated by \thebibliography
%% use before \bibliography or \bibliographytext 
\def\setbibindent#1{%
\let\oldthebibliography\thebibliography
\def\thebibliography##1{\oldthebibliography{#1}}
}



\catcode`\@=12

\long\def\skipover#1{}

\long\def\correction#1{\footnote{TO CORRECT: #1}}
\def\nocorrections{\long\def\correction##1{}}
\def\question#1{{\bf Question:} {\em #1} }
\long\def\todo#1{{\bf To Do:} {\em #1} }


\def\twiddle{$\sim$}

%% Macro for URLs
%% puts them in tt format and  breaks line after dot if needed.

\skipover{
\def\urlspace{-0.5em}
\def\url#1{{\tt \urlsub#1\end}}
%\def\urldot{. \hspace{-1em}\urlsub}
\def\urldot{. \linebreak[1]\hspace\urlspace\urlsub}
\def\urlcolon{: \linebreak[1]\hspace\urlspace\hspace\urlspace\urlsub}
\def\urlslash{/ \linebreak[1]\hspace\urlspace\urlsub}
\def\urltilde{\twiddle\urlsub}
\def\urlsub#1{\ifx#1\end \let\next=\relax
      \else \ifx#1.\let\next=\urldot
      \else \ifx#1:\let\next=\urlcolon
      \else \ifx#1/\let\next=\urlslash
      \else \ifx#1~\let\next=\urltilde
      \else#1\let\next=\urlsub\fi\fi\fi\fi\fi\next}
}

% Input a file in verbatim mode (eg a program file)
\def\verbinput#1{\expandafter\begin{verbatim}\input#1}
%NB: usage: \verbinput{filename}\end{verbatim}


\def\mb#1{{\mathbf{#1}}}
\def\mbf#1{{\mathbf{#1}}}


%%% MACROS for SLIDES


\def\heading#1{%
\centerline{\shadowbox{
\large \begin{tabular}{c} #1 \end{tabular}
           }  }
\vspace{1ex minus 1ex}
}

\def\bheading#1{%
\centerline{\shadowbox{
\large\bf \begin{tabular}{c} #1 \end{tabular}
           }  }
\vspace{1ex minus 1ex}
}

\def\printlandscape{\special{landscape}}    % Works with dvips.
\def\printportrait{\special{portrait}}

\def\leedslogo#1{\centerline{
\psfig{file=uol.eps,width=#1} %% May look wrong in xdvi
}}

\def\titleslide#1#2{
\begin{slide*}
~

\vfill
\heading{#1}

\vfill
\begin{center}
        {\large\bf Brandon Bennett} 

\vspace{3ex}
        Division of AI \\
        School of Computing \\
        University of Leeds\\ 
        Leeds LS2 9JT, England \\
        {\tt brandon@comp.leeds.ac.uk} \\
\end{center}

\vfill
\centerline{
\psfig{file=uol.eps,width=0.7in} %% May look wrong in xdvi
}
\vfill
\centerline{#2}
\vfill
\end{slide*}
}


\def\nlf{\\ \mbox{} \hfill}

%%\def\fixhyphens{\usepackage[english]{babel}}

%% FIX HYPHENATION
\lefthyphenmin=2
\righthyphenmin=3

%% special commands for spell checker
\def\sic#1{#1}
\def\sico!#1!{#1}
\def\sicname#1{#1}
\def\accept#1{}
\def\acceptname#1{}

%\newenvironment{nospell}{}{}
\def\spelloff{}
\def\spellon{}


%% Check whether a macro is defined
\def\ifundefined#1{\expandafter\ifx\csname#1\endcsname\relax}

%% Define \ifpdf
%% Used in graphic stuff to check if pdf is running
%% This is taken from ifpdf.sty
\newif\ifpdf
\ifx\pdfoutput\undefined
\else
  \ifx\pdfoutput\relax
  \else
    \ifcase\pdfoutput
    \else
      \pdftrue
    \fi
  \fi
\fi

\ifpdf
  \typeout{Running in PDF mode}
\else
  \typeout{Running in DVI mode}
\fi

%%%% CLEVER GRAPHIC INCLUSION

\def\pdfgraphictype{jpg} %% or perhaps pdf or png
\ifpdf   
  \def\graftype{\pdfgraphictype}  
\else
  \def\graftype{eps}
\fi


\typeout{* BB mymacros: Default graphic type set to: \graftype}

\ifpdf
   \def\optpdftex{pdftex}
   \def\optgraphic[#1]#2#3{\includegraphics[#1]{#3}} 
\else
   \def\pdfinfo#1{}
   \def\optpdftex{}
   \def\optgraphic[#1]#2#3{\includegraphics[#1]{#2}} 
\fi

\def\altgraphic[#1]#2#3#4{\optgraphic[#1]{#2.#3}{#2.#4}}

\def\figurepath{}
\def\graphswap[#1]#2{\optgraphic[#1]{\figurepath#2.eps}{\figurepath#2.\pdfgraphictype}}

%% \graphicifpdf includes graphic in pdf mode
%% otherwise just draws a box with the file name
%% I haven't really used this much.
\ifpdf
     \def\graphicifpdf[#1]#2{\includegraphics[#1]{#2}}
   \else
     \def\graphicifpdf[#1]#2{\fbox{\parbox{1in}{#1\\ #2}}}
\fi 

%\def\setaltgraphic#1#2{%
%   \def\altgraphic[##1]##2{\optgraphic[##1]{##2.#1}{##2.#1}}}

\def\graphic[#1]#2{\includegraphics[#1]{#2.\graftype}} 

\def\psfiggraphic{\renewcommand{\graphic}[2][]{\psfig{##1,file=##2.eps}}}

%%% Date functions

\def\dayth{\number\day\ifcase\day \or
st\or nd\or rd\or th\or th\or th\or th\or th\or th\or th\or 
th\or th\or th\or th\or th\or th\or th\or th\or th\or th\or
st\or nd\or rd\or th\or th\or th\or th\or th\or th\or th\or
st\fi}

\def\monthname{\ifcase\month \or
January\or February\or March\or April\or May\or June\or
July\or August\or September\or October\or November\or December\fi}


%%%% How to set up new font sizes
%% define a font which is Helvetica at 4.5 pt
\newfont{\boxfont}{hv at 4.5 pt}

% How to rename or alter the References section
% \renewcommand\refname{References\footnote{The   reference  list  given
%     here is somewhat incomplete due to limited preparation time. It is
%     intended  to  add several  more  recent  references  to the  final
%     version of this paper, relating the discussion to current works on
%     ontology      development.}}      


\typeout{** finished loading mymacros.tex **}

%%%%%%%%%%%%%%%%%%%%%%%%%%%%%%%%%%%%%%%%%%%%%%%%%%%%%%%%%%%%%%%
%%  END END END         (of mymacros.tex)      END END END   %%
%%%%%%%%%%%%%%%%%%%%%%%%%%%%%%%%%%%%%%%%%%%%%%%%%%%%%%%%%%%%%%%



\usepackage{makeidx}  % allows for indexgeneration
%
\usepackage{graphicx}         % standard LaTeX graphics tool
                              % when including figure files
\begin{document}
%
\frontmatter          % for the preliminaries

\title{Taking shortcuts with OWL using safe macros}

\author{Chris Mungall\inst{1}, Alan Ruttenberg\inst{2}, David Osumi-Sutherland\inst{3}}

\institute{
Genomics Division, Lawrence Berkeley National Laboratory, Berkeley, CA, USA\\
\email{CJMungall@lbl.gov}
\and
Science Commons, Cambridge, MA, USA \\
\email{alanruttenberg@gmail.com}
\and
Department of Genetics, University of Cambridge, Cambridge, UK \\
\email{djs93@gen.cam.ac.uk}
}

\maketitle              % typeset the title of the contribution

\begin{abstract}

Accurate representation of complex domains such as biology demands
powerful and expressive ontology languages such as OWL. However, the
complex nested class expressions required for modeling can be a
hindrance to ontology authoring and adoption. These class expressions
can appear opaque to domain experts, and even users proficient in OWL
can benefit from some kind of syntactic sugar or ``short-cut''
strategy, especially when authoring large ontologies.

One solution is to have domain experts fill in simple templates (for
example, in Excel) and translate the results into more complex axioms,
but this has the disadvantage of being disconnected from full ontology
authoring and reasoning environment.

We present here a method of specifying shortcut properties directly in
OWL. These shortcut properties can be used in similar ways as object
properties within the OWL environment, with the resulting simple
axioms translated automatically to more complex axioms via macro
expansion. We describe some example scenarios where this is of use in
authoring existing bio-ontologies.

One of the main implications of this work is a way to simplify the
translation between OBO format and OWL, and the use of RDF
triple-stores with complex OWL ontologies.

\end{abstract}

\section{Introduction}

% TODO: 
%  number axioms /ref

Accurate representation of complex domains such as biology demand
powerful and expressive ontology languages such as OWL. However, the
very expressive power of OWL can be a hindrance to widespread adoption
and effective use - domain experts may not be comfortable with complex
axioms and deeply nested class expressions. Even those proficient in
OWL may prefer to work in terms of high-level templates that abstract
away from the logical ``machine code'' of OWL. This is particularly
true for the development of large ontologies, which often requires the
repeated application of the same template pattern for multiple
classes. In addition to being tedious and error-prone, the high-level
pattern is not immediately apparent on viewing the ontology. Table
\ref{tab:example-templates} shows some example repeated patterns
extracted from core bio-ontologies from the OBO Foundry
registry\cite{Smith2007}.

  \begin{table}
    \begin{tabular}{ | p{1.1cm} | p{4cm} | p{7cm} | }
      \hline 
      \textbf{Source} & \textbf{Pattern} & \textbf{Example} \\
      \hline

      DC-CL &

      hasPart some ('plasma membrane' and hasPart $?Y$)

      &

      Class: 'alpha-beta T cell' \newline
      EquivalentTo: 'T cell' and 
      hasPart some ('plasma membrane' and hasPart 'alpha-beta TCR complex')

      \\

      \hline

      CL &

      bearerOf some (realizedBy only $?Y$)

      &

      Class: 'immune cell' \newline
      EquivalentTo: 'cell' and bearerOf some (realizedBy only 'immune system process')

      \\

      \hline

      GO &

      (partOf some $?X$) disjointFrom (partOf some $?Y$)

      &

      (partOf some nucleus) disjointFrom (partOf some cytoplasm)

      \\

      \hline

      Fly &

      Class: $?X$ \newline SubClassOf: partOf some $?Y$ \newline 
      Class: $?Y$ \newline SubClassOf: hasPart some $?X$

      &

      Class: retina \newline
      SubClassOf: partOf some eye \newline

      Class: eye \newline
      SubClassOf: hasPart some retina 
      \\

      \hline

      Fly &

      hasPart some ('presynaptic membrane' and partOf some (synpase
      and hasPart some ('postsynaptic membrane' and partOf some ?Y)))

      &

      Class: 'giant fiber neuron' \newline SubClassOf: hasPart some
      ('presynaptic membrane' and partOf some (synpase and hasPart
      some ('postsynaptic membrane' and partOf some 'tergotrochanteral
      muscle motor neuron')))

      \\

      \hline

      CL &

      hasPart exactly 0 $?Y$

      &

      Class: 'enucleate cell' \newline
      EquivalentTo: cell and hasPart exactly 0 nucleus

      \\

      \hline

      TAO &

      hasPart some (partOf some $?Y$)

      &

      Class: 'vertebra 5' \newline
      SubClassOf: hasPart some (partOf some 'V4-5 joint')

      \\

      \hline

      OBI &

      realizes some ('analyte role' and roleOf some ('scattered molecular aggregate' and hasGrain only $?Y$))

      &

      Class: 'human antithrombin-III (AT-III) in blood assay' \newline
      SubClassOf: realizes some ('analyte role' and (roleOf some 
      ('scattered molecular aggregate' and (hasGrain some 'human antithrombin-III protein'))))

      \\

      \hline
    \end{tabular}
    \caption{Example generic patterns taken from various ontologies in
      the OBO library (http://obofoundry.org). CL = cell ontology,
      DC-CL = dendritic cell ontology, Fly = \emph{Drosophila}
      anatomy, GO = Gene Ontology, OBI = Ontology for Biomedical
      Investigations, TAO = Teleost anatomy ontology. Note that for GO
      and CL these axioms are available as optional
      extensions\cite{Mungall_GOXP_2010}\cite{Diehl_ICBO_2009}}.
    \label{tab:example-templates}
  \end{table}

There are a number of approaches to this problem. Quick Term Templates
(QTTs)\cite{QTT2009} are a way of generating complex OWL axioms from
simple tables. The domain expert can use a familiar tool such as
Excel, structured according to a predetermined template, rather than
working directly in an ontology development environment such as
Protege. A similar approach is to defined a domain-specific
intermediate representation language, and translate this to OWL.

However, this approach can be unsatisfying for a number of reasons. A
scripting approach is ad-hoc and difficult to integrate with existing
OWL toolchains. In addition, the source intermediate representation is
lost on translation into OWL -- ideally the intermediate representation
would still be visible within environments and tools such as
Protege and Pellet.

%The
%ontology consumer (who may be typically be a domain scientist rather
%than a logician) also has to view the ontology at this ``low level''.

%However, for many users
%it may be preferable to work with the ontology at the template level.

Here we describe a macro expansion language that allows the
representation in a high-level intermediate form directly in OWL,
through the use of high-level shortcut relations. A generic simple
macro-expansion tool can be used to expand these shortcut relations
into more complex axioms and class-expressions. This has the advantage
of allowing the intermediate representation to be viewed and
manipulated from within standard OWL tools and environments.

%allows users to
%view and edit at either level, as appropriate.

%This representation has the advantage that the ``sweetened'' form can
%be directly represented in OWL-DL.

%We also discuss the usage of a macro expansion language versus using
%and extending existing OWL2 constructs such as property chains.

\section{Macro Expansion Language}

We propose a macro expansion language that can be represented directly
within OWL-DL. The macro expansions are represented as OWL Manchester
Syntax\cite{Horridge2006} with the addition of template
variables. Either annotation properties or object properties are
annotated with macros, and these can be used to expand individual
axioms at any time during the ontology development or deployment
cycle, preferably prior to reasoning. We refer to properties that are
annotated in this way as \emph{shortcut relations}.  Table
\ref{tab:macro-expansion} shows the two different annotation
properties that can be used to define shortcut relations. The first is
intended to apply to annotation properties connecting two classes, and
the second is intended to apply to object properties used in different
contexts. In both cases the value of the annotation property is a
template string containing Manchester syntax with variable
placeholders denoted by the $?$ symbol.

The pattern in the second row of table \ref{tab:example-templates}
could be specified as follows:

\begin{verbatim}
AnnotationProperty: capableOf
Annotations: expandExpressionTo "bearerOf some (realizedBy only ?Y)"
\end{verbatim}

  \begin{table}
    \begin{tabular}{ | p{3.2cm} | p{4.5cm} | p{4.5cm} | }
      \hline 
      \textbf{Annotation Property} & \textbf{Usage} & \textbf{Matches} \\

      \hline
      expandAssertionTo & Expanding annotation assertion axioms connecting two classes & PropertyAssertion(?R,$?X$,$?Y$) \\

      \hline
      expandExpressionTo & Expanding class expressions & 
      $?R$ some $?Y$ \textbf{or} \newline
      $?R$ value $?Y$ \\

      \hline
    \end{tabular}
    \caption{Annotation properties use to specify a macro
      expansion. If a property $?R$ is annotated using one of these
      two annotation properties, then the ontology is scanned for
      occurrences of the match pattern, which is then expanded
      according to the specified template.}
    \label{tab:macro-expansion}
  \end{table}

We now describe these in more detail, first describing expansion of
property assertions, followed by expansions of expressions.

\subsection{Macro Expansion of Annotation Properties}

We propose an annotation property \emph{expandAssertionTo} that
specifies how to expand a single OWL AnnotationPropertyAssertion into a more
complex axiom by substituting two variables $?X$ and $?Y$ with the
subject and target of the assertion respectively.

The rule can be specified as:

\begin{exprule}\label{exp-pa}
\pr{PropertyAssertion}(expandAssertionTo,?R,?T)  $\Rightarrow$ \newline
 substitute(\pr{PropertyAssertion}(?R,?X,?Y),?T)
\end{exprule}

Where \emph{substitute} is a two-argument function that replaces
axioms or expressions (argument 1) with a template (argument 2).

For example, given the following shortcut relation specification:

\begin{verbatim}
AnnotationProperty: spatially_disconnected_from
Annotations: expandAssertionTo 
  "DisjointClasses: (part_of some ?X)  (part_of some ?Y)"
\end{verbatim}

The following annotation property assertion axiom:

\begin{verbatim}
Class: nucleus
Annotations: spatially_disconnected_from cytoplasm
\end{verbatim}

would expand to a general class inclusion (GCI) axiom:

\begin{verbatim}
DisjointClasses: (part_of some nucleus) disjointFrom (part_of some cytoplasm)
\end{verbatim}

Note that this pattern is not applicable to all cases in table
\ref{tab:example-templates} - for example, we may want to define a
class using an equivalence axiom. Here we want a generic way of
expanding class expressions using shortcut relations, such that they
can be used in different axiom types. This is where the second type of
expansion comes in.

\subsection{Macro Expansion of Object Properties}

As an alternative that works for both SubClass and EquivalentTo
axioms, we propose a different expansion rule for a new annotation
property expandExpressionTo. Here, the macro describes how to expand a class
expression, not a property assertion, and only uses a single variable.

The macro object property should be used in existential restrictions,
and the translation rule is:

\begin{exprule}\label{exp-some}
\pr{PropertyAssertion}(expandExpressionTo,?R,?T)  $\Rightarrow$ \newline
substitute(?R some ?Y,?T)
\end{exprule}

For example, the cell ontology defines a shortcut relation used for
defining cell types based on the proteins expressed on the plasma
membrane (i.e. the boundary of the cell)\cite{Masci2009}.

\begin{verbatim}
ObjectProperty: has_plasma_membrane_part
Annotations: expandExpressionTo 
  "has_part some ('plasma membrane' and has_part some ?Y)"
\end{verbatim}

This allows us to specify axioms using existential restrictions as follows:

\begin{verbatim}
Class: 'alpha-beta T cell'
EquivalentTo: 'T cell' and has_plasma_membrane_part some ('alpha-beta TCR complex')
\end{verbatim}

This intermediate level representation would expand to:

\begin{verbatim}
Class: 'alpha-beta T cell'
EquivalentTo: 'T cell' and 
  has_part some ('plasma membrane' and 
                 has_part some 'alpha-beta TCR complex')
\end{verbatim}

Note that the intermediate level representation can still be used by a
reasoner. Whilst it may fail to make all the correct inferences, in
this particular case it will not produce any incorrect inferences. We
regard this semantically correct intermediate representation as a
desirable trait, even if the intention is to ultimately reason over
the expanded form.

However, if existential restrictions are used in the intermediate
level representation, then it is possible to have an intermediate
representation that is semantically incompatible with the expanded
form.

Consider the following shortcut relation:

\begin{verbatim}
ObjectProperty: lacksPart
Annotations: expandExpressionTo "hasPart exactly 0 ?Y"
\end{verbatim}

The author would write in the intermediate form:

\begin{verbatim}
Class: 'enucleate cell'
EquivalentTo: cell and lacks_part some nucleus
\end{verbatim}

Which would expand to:

\begin{verbatim}
Class: 'enucleate cell'
EquivalentTo: cell and has_part exactly 0 nucleus
\end{verbatim}

% DOS:Curious why you prefer cardinality approach rather than negation as Rob uses.
% CJM: makes no difference

Note that the intermediate form would produce \emph{different}
inferences from the expanded form. For example, if nucleus is a
subclass of organelle, then the intermediate form entails that a
enucleate cell lacks part some organelle. When expanded, this becomes
too strong.

Notice that table \ref{tab:macro-expansion} has an additional rule for
hasValue restrictions:

\begin{exprule}\label{exp-some}
\pr{PropertyAssertion}(expandExpressionTo,?R,?T)  $\Rightarrow$ \newline
substitute(?R value ?Y,?T)
\end{exprule}

Here we are treating the intermediate property as a relation between
an individual and a class.

Now we can write an intermediate representation:

\begin{verbatim}
Class: 'enucleate cell'
EquivalentTo: cell and lacks_part value nucleus
\end{verbatim}

% I'd like a little more explanation of what this actually means.
% Also note - this may be legal OWL 2, but there are tooling issues.  P4 doesn't like this.  We should certainly test this against major reasoners.

This expands to the correct form. Whilst the intermediate form does
not produce all the same inferences as the intermediate form, the
intermediate form does not produce any incorrect inferences.

% DOS: So, the idea would be to use this value type expansions for these dangerous cases only?
% CJM: yes - or perhaps for *all* cases, to be consistent

The meaning of the intermediate form may appear curious.  Here we are
treating the class \emph{nucleus} as part of the domain of discourse,
which is allowed in OWL2. The lacksPart shortcut relation holds between
instances  and classes.

We can also safely introduce an expansion rule for individuals:

\begin{exprule}\label{exp-inst}
\pr{expandExpressionTo}(?R,?T)  $\Rightarrow$ \newline
substitute(PropertyValue(?R,?X,?Y),?X Types: substitute(?R,?Y))
\end{exprule}

We can write property assertions:

\begin{verbatim}
Individual: imaged-cell00001234
Properties: lacks_part nucleus
\end{verbatim}

Which expand to:

\begin{verbatim}
Individual: imaged-cell00001234
Types: has_part exactly 0 nucleus
\end{verbatim}

\section{Discussion}

\subsection{Comparison with property chains}

Property chains are a powerful feature introduced into OWL2. They are
even more powerful when combined with self-restrictions, and can be
used to recapitulate some of the features of the macro-expansion rules
described above. For example, the \emph{has plasma membrane part}
relation could be specified as follows:

\begin{verbatim}
ObjectProperty: has_plasma_membrane_part 
SubPropertyChain: has_part o is_plasma_membrane o has_part

ObjectProperty: is_plasma_membrane

Class: 'plasma membrane'
SubClassOf: is_plasma_membrane Self
\end{verbatim}

A minor disadvantage here is that opaque axioms and properties are
inserted into the ontology -- but these could in principle be hidden
given appropriate annotation properties and tooling.

A more serious disadvantage is that property chains cannot be used to
\emph{define} a property. The direction of implication is one-way
only.

To see this, consider a standard axiom in mereology defining an
overlaps relation in terms of parthood:

$$
\pr{overlaps}(x,y) \dimp \E z: \pr{hasPart}(x,z) \con \pr{partOf}(z,y)
$$

The overlaps relation is useful in bio-ontologies\cite{dahdul2009}, so
it can be tempting to declare an object property for it, and provide a
property chain:

\begin{verbatim}
ObjectProperty: overlaps
SubPropertyChain: has_part o part_of
\end{verbatim}

But this axiom is weaker, and there is a danger if this property is
used in assertions thinking it has the same semantics. If an ontology
asserts that a overlaps some b, then this is \emph{not} the same as
asserting that a hasPart some partOf some b. If a user queries for
\emph{hasPart some partOf some b}, then classes asserted to be
subclasses of \emph{overlaps some b} will \emph{not} be returned.

The guiding principle here is: if the intention is to \emph{define} a
property then a shortcut relation should be specified. If the
intention is to infer property assertions then use a property chain.

Note that property chains cannot be used for shortcut relations such
as \emph{capableOf}, because in this case the shortcut relation
abbreviates a nested class expression involving both an existential
and universal restriction.

\subsection{Practical usage and implementation: ontology views}

The scheme we have proposed is relatively simple and can be
implemented easily.

However, to be maximally useful, the translator could be integrated
into ontology editing environments such as
Protege4\cite{horridge2006supporting}. Expanded axioms could be marked
using axiom annotations. Users would have the option of viewing the
ontology at the intermediate level, the expanded level, or
both. Changes at one level would automatically expand/contract to the
other level, and feed in to incremental reasoning.

However, even without this tight integration we believe that a
translation tool implementing the macro language specified here would
be useful.

There is a possible concern about reasoning over the intermediate
view, prior to expansion. However, our strategy here insulates us from
making incorrect inferences at the intermediate level. Users will
still have to be aware of the differences between the two levels to
know why they get more inferences in the expanded view.

% interestingly, we could write axioms specific to the intermediate
% view, effectively making it more expressive in some ways than the
% expanded view. But let's keep it simple and avoid this for now.

\subsection{Applications for OBO to OWL translation}

OBO-Edit is an ontology editing environment popular amongst
biologists\cite{Day-Richter2007}. The native model for this tool is
the OBO file format, which can be translated to a subset of
OWL\cite{golbreich2007obo}\cite{tirmizi2009}. A characteristic feature
of the OBO format is that an ontology is treated as a labeled graph,
much like RDF. Edges in the graph are by default translated to
existential restrictions. Whilst it is possible to write nested class
expressions in OBO format by using RDF b-node style anonymous classes,
this is not well-supported and best avoided. In addition, there is no
standard way to write other OWL constructs such as negation and
universal restriction, which are becoming increasingly necessary for
bio-ontologies.

Many OBO-Edit authored ontologies have started informally adopting
shortcut relations with a view to translating these to expanded OWL
axioms further down the line. One approach is to do this as part of
the obo to owl translation, but we believe there are advantages to
doing the expansion as a separate independent step. For one thing, it
simplifies the already complicated translation, which has some
problems in the existing translation. It also has the advantage of
maintaining the shortcut relations in the OWL environment.

Currently a piece of obo-format may look like this:

\begin{verbatim}
[Typedef]
id: ro:has_part
is_transitive: true
inverse_of: ro:part_of

[Typedef]
id: lacks_part
expandExpressionTo: "ro:has_part exactly 0 ?Y" []

[Term]
id: cl:enucleate_cell
intersection_of: cl:cell
intersection_of: lacks_part go:nucleus
\end{verbatim}

Using a simple extension of the standard obo-to-owl specification, this
would translate to:

\begin{verbatim}
ObjectProperty: lacks_part
Annotations:
  expandExpressionTo "ro:has_part exactly 0 ?Y"

Class: cl:enucleate_cell
EquivalentTo: cl:cell and lacks_part value go:nucleus
\end{verbatim}

Applying the expansions rules we would get:

\begin{verbatim}
ObjectProperty: ro:has_part
Characteristics: Transitive

ObjectProperty: lacks_part
Annotations:
  expandExpressionTo "ro:has_part exactly 0 ?Y"

Class: cl:enucleate_cell
EquivalentTo: cl:cell and ro:has_part exactly 0 go:nucleus
\end{verbatim}

Which has the intended semantics.

This has the advantage of allowing the users of OBO format tools to
access the more expressive constructs in OWL, yet retain a simple
graph-like model. This has positive implications for many downstream
consumers of OBO format ontologies, many of which are tools or
databases that also have a simple graph-like model of ontologies.

\subsection{Applications for RDF and semantic web applications}

If an OWL ontology contains complex axioms involving nested class
expressions this can result in RDF triples that are difficult to work
with at the RDF level. This difficulty is ameliorated by languages
such as SPARQL-DL, but there is still problem for many triple stores
and RDF libraries that are not OWL-aware.

The shortcut relation strategy provided here could provide a means of
providing a ``triple-friendly'' view over complex OWL ontologies.

\section{Availability and future directions}

A demo implementation of the macro-expansion system is available as
part of Thea2\cite{Vangelis_2009}. Futher examples in obo and owl are
available from the obo2owl
repository\footnote{http://github.org/cmungall/obo2owl}.

Future extensions planned include support for n-ary relations and
description graphs.

\section{Conclusions}

We have defined two annotation properties that can be used to specify
shortcut relations that can be used in intermediate OWL-DL
representations that provide a simplified view over a complex
ontology. Axioms or expressions that use these relations can be
expended to a more verbose form prior to reasoning. At the same time,
the intermediate form is safe for reasoning.

We believe that these shortcut relations will assist with ontology
authoring for both novices and experts. We also believe that these
relations will assist with interconversion between OWL and popular but
less expressive formalisms and data models, opening up OWL to a wider
community of users.

\section{References}

\bibliography{owlmacros}
\bibliographystyle{plain}

\end{document}

